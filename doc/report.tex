\documentclass[12pt]{article}
\usepackage{amsmath,amssymb,amsthm, minted, hyperref, enumerate, graphics, graphicx}
\usepackage[margin=1in]{geometry}
\usepackage{listings}
\usepackage{color}

\definecolor{dkgreen}{rgb}{0,0.6,0}
\definecolor{gray}{rgb}{0.5,0.5,0.5}
\definecolor{mauve}{rgb}{0.58,0,0.82}

\lstset{frame=tb,
  language=Java,
  aboveskip=3mm,
  belowskip=3mm,
  showstringspaces=false,
  columns=flexible,
  basicstyle={\small\ttfamily},
  numbers=none,
  numberstyle=\tiny\color{gray},
  keywordstyle=\color{blue},
  commentstyle=\color{dkgreen},
  stringstyle=\color{mauve},
  breaklines=true,
  breakatwhitespace=true,
  tabsize=3
}

\title{Loom Proof of Concept}
\author{Mahesh Khanwalkar}
\date{}

\begin{document}
    \maketitle
    \section{Introduction}

    Project Loom is an effort to add scoped, one-shot delimited continuations to Java, along with the
    necessary runtime support within the JVM. Third-party implementations of delimited continuations, like
    Kilim required an explicit, after-compilation weaving process; however, this is not necessary within
    Loom.

    \subsection{Basic API}

    The way of creating continuations is through the aptly-named Continuation class, which encapsulates a
    Runnable target. The continuation can be started for the first time using a call to the run method:

    \begin{lstlisting}
    // Runnable body is wrapped within the continuation
    Continuation cont = new Continuation(body); 
    cont.run();
    \end{lstlisting}

    This call performs all the necessary setup required, which includes the support for handling yielding.
    Within the Runnable body, the continuation can be yielded using a static call to Continuation.yield,
    which will cause the call to run to return. However, the same condition is true when the continuation
    simplify finishes. Therefore, it is up to the caller to check whether the continuation actually finished.
    It is up to the user of the continuation to resume it, which can be done using the same run call.

    \begin{lstlisting}
    if(cont.isDone()) {
        // The continuation has finished
    } else {
        // The continuation has yielded (suspended)
        // Calling run will resume right after the point of suspension.
        cont.run();
    }
    \end{lstlisting}

    \subsection{Limitations}
    The API is still currently under active development, so there are still some limitations. The only
    exposed state of a continuation is whether it is done or not. There is no way to tell from another thread
    that the continuation has yielded.

    The documentation mentions a Fiber class, which would manage the scheduling of continuations and states
    that continuations are not necessarily meant to be used directly, since they are quite low-level.
    However, there seems to be a gap between the documentation and the current state of the code base. 

    There was a Fiber class in an earlier build, but it was marked package private, so it is quite likely
    that it was not supposed to be used directly. In addition, in newer builds, it looks like the class has
    been removed completely. The only public, relatively-stable interface is the Continuation class. 

    Since Project Loom is under active development, it is natural that there are still many bugs. JVM crashes
    do occur occasionally, so some of the runtime support does seem to have some issues that need to be
    resolved. Debugging also occasionally causes the JVM to crash as well which is not ideal. However, for
    the most part, the current state of Loom is stable enough and feature-ready enough to be used to
    implement higher level parallelism constructs.
\end{document}
